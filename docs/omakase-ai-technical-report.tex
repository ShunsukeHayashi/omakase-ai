\documentclass[a4paper,11pt]{article}

% ===== パッケージ =====
\usepackage[utf8]{inputenc}
\usepackage[T1]{fontenc}
\usepackage{lmodern}
\usepackage[english,japanese]{babel}
\usepackage{xeCJK}
\usepackage{graphicx}
\usepackage{geometry}
\usepackage{hyperref}
\usepackage{xcolor}
\usepackage{listings}
\usepackage{fancyhdr}
\usepackage{titlesec}
\usepackage{booktabs}
\usepackage{longtable}
\usepackage{array}
\usepackage{float}

% ===== ページ設定 =====
\geometry{
  left=25mm,
  right=25mm,
  top=30mm,
  bottom=30mm
}

% ===== 色定義 =====
\definecolor{primary}{RGB}{33, 37, 41}
\definecolor{secondary}{RGB}{108, 117, 125}
\definecolor{accent}{RGB}{0, 123, 255}
\definecolor{codebg}{RGB}{248, 249, 250}

% ===== リンク設定 =====
\hypersetup{
  colorlinks=true,
  linkcolor=accent,
  urlcolor=accent,
  citecolor=accent
}

% ===== コードブロック設定 =====
\lstset{
  backgroundcolor=\color{codebg},
  basicstyle=\ttfamily\small,
  breaklines=true,
  frame=single,
  rulecolor=\color{secondary},
  numbers=left,
  numberstyle=\tiny\color{secondary},
  keywordstyle=\color{accent},
  commentstyle=\color{secondary},
  stringstyle=\color{primary}
}

% ===== ヘッダー・フッター =====
\pagestyle{fancy}
\fancyhf{}
\fancyhead[L]{\textcolor{secondary}{\small omakase.ai Technical Report}}
\fancyhead[R]{\textcolor{secondary}{\small 2025-12-06}}
\fancyfoot[C]{\textcolor{secondary}{\thepage}}
\renewcommand{\headrulewidth}{0.4pt}
\renewcommand{\footrulewidth}{0pt}

% ===== セクション書式 =====
\titleformat{\section}
  {\Large\bfseries\color{primary}}
  {\thesection}{1em}{}
\titleformat{\subsection}
  {\large\bfseries\color{primary}}
  {\thesubsection}{1em}{}
\titleformat{\subsubsection}
  {\normalsize\bfseries\color{primary}}
  {\thesubsubsection}{1em}{}

% ===== 日本語フォント =====
\setCJKmainfont{Hiragino Mincho ProN}
\setCJKsansfont{Hiragino Kaku Gothic ProN}

% ===== ドキュメント開始 =====
\begin{document}

% ===== 表紙 =====
\begin{titlepage}
  \centering
  \vspace*{3cm}

  {\Huge\bfseries\color{primary} omakase.ai\\[0.5cm] Technical Analysis Report}

  \vspace{1cm}

  {\Large\color{secondary} 技術分析ドキュメント}

  \vspace{2cm}

  {\large Version 1.0}

  \vspace{0.5cm}

  {\large 2025年12月6日}

  \vfill

  {\normalsize
    HAR解析に基づくVAPI/Daily.coプロトコル分析\\
    競合比較・技術スタック評価
  }

  \vspace{2cm}

  {\small\color{secondary} Confidential - Internal Use Only}
\end{titlepage}

% ===== 目次 =====
\tableofcontents
\newpage

% ===== 1. エグゼクティブサマリー =====
\section{エグゼクティブサマリー}

omakase.aiはEC(電子商取引)サイト向けの音声AIショッピングアシスタントプラットフォームである。
本レポートでは、プロダクションHARファイルの解析に基づき、技術スタック、外部依存関係、独自開発部分を詳細に分析する。

\subsection{主要な発見}

\begin{itemize}
  \item \textbf{音声AI基盤}: VAPI(Voice AI Platform)に完全依存
  \item \textbf{通信基盤}: Daily.coのWebRTC SFUを使用
  \item \textbf{認証}: Clerkによるセッション管理
  \item \textbf{独自開発}: Widget UI、プロンプト設計、バックエンド統合
\end{itemize}

\subsection{技術依存度}

\begin{table}[H]
  \centering
  \begin{tabular}{lcc}
    \toprule
    \textbf{コンポーネント} & \textbf{提供元} & \textbf{依存度} \\
    \midrule
    音声認識 (STT) & VAPI/Deepgram & 高 \\
    音声合成 (TTS) & VAPI/ElevenLabs & 高 \\
    LLM & VAPI/OpenAI/Claude & 高 \\
    WebRTC & Daily.co & 高 \\
    認証 & Clerk & 中 \\
    Widget UI & 独自開発 & - \\
    バックエンドAPI & 独自開発 & - \\
    \bottomrule
  \end{tabular}
  \caption{技術依存マトリクス}
\end{table}

\newpage

% ===== 2. システムアーキテクチャ =====
\section{システムアーキテクチャ}

\subsection{全体構成}

omakase.aiは以下の4層構造で構成される:

\begin{enumerate}
  \item \textbf{Presentation Layer}: React Widget
  \item \textbf{Transport Layer}: Daily.co WebRTC
  \item \textbf{AI Platform Layer}: VAPI
  \item \textbf{Application Layer}: Express.js Backend
\end{enumerate}

\subsection{アーキテクチャ図}

\begin{figure}[H]
  \centering
  \includegraphics[width=0.95\textwidth]{images/omakase-ai-architecture.png}
  \caption{omakase.ai システムアーキテクチャ}
\end{figure}

\subsection{データフロー}

\begin{enumerate}
  \item ユーザーがWidgetで通話開始をクリック
  \item ClerkでセッションJWTを検証
  \item VAPIに通話開始リクエスト(assistantId + page\_context)
  \item Daily.coでWebRTCルーム作成・接続
  \item 音声ストリームがVapi Listener/Speakerと双方向通信
  \item VAPIがFunction Toolsでバックエンドを呼び出し
\end{enumerate}

\newpage

% ===== 3. プロトコル詳細 =====
\section{音声通話プロトコル}

HAR解析により、VAPI/Daily.coの詳細なプロトコルフローを特定した。

\subsection{通話開始シーケンス}

\begin{figure}[H]
  \centering
  \includegraphics[width=0.98\textwidth]{images/protocol.png}
  \caption{VAPI/Daily.co 音声通話プロトコルシーケンス}
\end{figure}

\subsection{主要フェーズ}

\begin{longtable}{lp{10cm}}
  \toprule
  \textbf{フェーズ} & \textbf{説明} \\
  \midrule
  \endhead

  認証 & Clerkで\texttt{POST /sessions/\{id\}/touch}を呼び出し、セッション検証 \\
  \midrule
  VAPI初期化 & \texttt{POST /call/web}でassistantIdとpage\_contextを送信 \\
  \midrule
  Daily.co接続 & \texttt{POST /rooms/check/vapi/\{roomId\}}でICE設定取得 \\
  \midrule
  WebSocket & SFUへのWebSocket接続、\texttt{join-for-sig}メッセージ \\
  \midrule
  WebRTC & \texttt{create-transport}、\texttt{connect-transport}、\texttt{send-track} \\
  \midrule
  エージェント参加 & Vapi Speaker/Listenerが\texttt{sig-presence}で参加 \\
  \midrule
  音声会話 & RTPパケットの双方向ストリーミング \\
  \bottomrule
\end{longtable}

\subsection{技術仕様}

\begin{itemize}
  \item \textbf{プロトコル}: WebSocket + WebRTC
  \item \textbf{トポロジー}: SFU (Selective Forwarding Unit)
  \item \textbf{コーデック}: audio/opus, 48kHz
  \item \textbf{ICE}: Cloudflare STUN, Twilio TURN
  \item \textbf{参加者}: User, Vapi Speaker, Vapi Listener
\end{itemize}

\newpage

% ===== 4. 外部サービス分析 =====
\section{外部サービス依存分析}

\subsection{VAPI (Voice AI Platform)}

\begin{table}[H]
  \centering
  \begin{tabular}{ll}
    \toprule
    \textbf{項目} & \textbf{詳細} \\
    \midrule
    エンドポイント & \texttt{https://api.vapi.ai/call/web} \\
    assistantId & \texttt{91eb9aaa-17dc-4aa0-862b-e28fa21c0df4} \\
    機能 & STT, TTS, LLM, Function Calling \\
    レイテンシ & 500ms未満 \\
    処理実績 & 150M+通話 \\
    \bottomrule
  \end{tabular}
  \caption{VAPI技術仕様}
\end{table}

\subsubsection{リクエスト例}

\begin{lstlisting}[language=json]
POST /call/web
{
  "assistantId": "91eb9aaa-...",
  "assistantOverrides": {
    "variableValues": {
      "page_context": {
        "product_name": "...",
        "price": 2500,
        "stock": "available"
      }
    }
  }
}
\end{lstlisting}

\subsection{Daily.co (WebRTC Infrastructure)}

\begin{table}[H]
  \centering
  \begin{tabular}{ll}
    \toprule
    \textbf{項目} & \textbf{詳細} \\
    \midrule
    シグナリング & \texttt{https://gs.daily.co/} \\
    SFU WebSocket & \texttt{wss://ip-xxx.wss.daily.co} \\
    グローバル拠点 & 75+ \\
    レイテンシ & 13ms中央値 \\
    SDK & daily-js v0.85.0 \\
    \bottomrule
  \end{tabular}
  \caption{Daily.co技術仕様}
\end{table}

\subsection{Clerk (Authentication)}

\begin{table}[H]
  \centering
  \begin{tabular}{ll}
    \toprule
    \textbf{項目} & \textbf{詳細} \\
    \midrule
    エンドポイント & \texttt{https://clerk.omakase.ai/} \\
    機能 & セッション管理, JWT発行 \\
    トークン更新 & 約45秒間隔 \\
    \bottomrule
  \end{tabular}
  \caption{Clerk技術仕様}
\end{table}

\newpage

% ===== 5. 独自開発部分 =====
\section{独自開発部分の分析}

\subsection{価値提供マトリクス}

\begin{table}[H]
  \centering
  \begin{tabular}{lccl}
    \toprule
    \textbf{コンポーネント} & \textbf{独自性} & \textbf{差別化} & \textbf{説明} \\
    \midrule
    プロンプト設計 & 高 & 高 & EC販売特化シナリオ \\
    Widget UI & 高 & 中 & ブランディング対応 \\
    バックエンドAPI & 中 & 高 & 商品/カート連携 \\
    Function Tools & 中 & 高 & カート追加等 \\
    認証統合 & 低 & 低 & Clerk使用 \\
    音声AI基盤 & 低 & 低 & VAPI依存 \\
    \bottomrule
  \end{tabular}
  \caption{独自開発価値マトリクス}
\end{table}

\subsection{独自開発ファイル構造}

\begin{lstlisting}[language=bash]
frontend/src/
  components/
    preview-phone.tsx      # プレビュー表示
    agent-selector.tsx     # エージェント選択
    config-panel.tsx       # 設定パネル
    ec-context-form.tsx    # EC連携設定
  lib/
    api.ts                 # APIクライアント
    realtime.ts            # WebSocket処理

src/server/
  routes/
    products.ts            # 商品API
    voice.ts               # 音声API
    knowledge.ts           # ナレッジAPI
  services/
    store.ts               # データストア
    prompt-generator.ts    # プロンプト生成
\end{lstlisting}

\newpage

% ===== 6. 競合分析 =====
\section{競合分析}

\subsection{主要競合比較}

\begin{table}[H]
  \centering
  \begin{tabular}{lccccc}
    \toprule
    \textbf{機能} & \textbf{omakase.ai} & \textbf{VAPI} & \textbf{Pipecat} & \textbf{Rep AI} & \textbf{Retell} \\
    \midrule
    音声対話 & \checkmark & \checkmark & \checkmark & \checkmark & \checkmark \\
    Webウィジェット & \checkmark & \checkmark & - & \checkmark & - \\
    EC特化 & \checkmark & - & - & \checkmark & - \\
    日本語 & \checkmark & \checkmark & \checkmark & ? & \checkmark \\
    カート統合 & \checkmark & - & - & \checkmark & - \\
    OSS & - & - & \checkmark & - & - \\
    \bottomrule
  \end{tabular}
  \caption{競合機能比較}
\end{table}

\subsection{ポジショニング}

omakase.aiは「EC特化」×「日本市場」の垂直特化ポジションを取る。
汎用プラットフォーム(VAPI, Pipecat)との差別化は、EC販売に最適化された会話設計とカート統合にある。

\subsection{Pipecat移行の可能性}

\begin{table}[H]
  \centering
  \begin{tabular}{ll}
    \toprule
    \textbf{メリット} & \textbf{デメリット} \\
    \midrule
    VAPI料金削減 & 開発工数増加 \\
    カスタマイズ自由度 & インフラ管理必要 \\
    ベンダーロックイン回避 & 保守負担増 \\
    \bottomrule
  \end{tabular}
  \caption{Pipecat移行の評価}
\end{table}

\newpage

% ===== 7. 推奨事項 =====
\section{推奨事項}

\subsection{短期 (3-6ヶ月)}

\begin{enumerate}
  \item \textbf{プロンプト・シナリオ強化}: 業種別テンプレート、購買心理考慮
  \item \textbf{分析ダッシュボード}: 会話分析、コンバージョン追跡
  \item \textbf{A/Bテスト基盤}: プロンプト・声・フローの最適化
\end{enumerate}

\subsection{中期 (6-12ヶ月)}

\begin{enumerate}
  \item \textbf{Pipecat検証}: PoC実施、コスト比較
  \item \textbf{データ蓄積}: 会話ログからの学習、レコメンド改善
  \item \textbf{データベース導入}: In-Memory → PostgreSQL + Redis
\end{enumerate}

\subsection{長期 (1-2年)}

\begin{enumerate}
  \item \textbf{自社音声AI}: Fine-tunedモデル、独自STT/TTS
  \item \textbf{プラットフォーム化}: SDK公開、マーケットプレイス
\end{enumerate}

\newpage

% ===== 8. 付録 =====
\section{付録}

\subsection{参考ドキュメント}

\begin{itemize}
  \item \texttt{docs/API.md} - API仕様書
  \item \texttt{docs/omakase-ai-protocol.puml} - プロトコルシーケンス
  \item \texttt{docs/architecture.puml} - アーキテクチャ図
  \item \texttt{docs/TECHNICAL\_ANALYSIS.md} - 技術分析詳細
  \item \texttt{docs/COMPETITIVE\_ANALYSIS.md} - 競合分析詳細
\end{itemize}

\subsection{外部リンク}

\begin{itemize}
  \item VAPI: \url{https://vapi.ai/}
  \item Daily.co: \url{https://www.daily.co/}
  \item Pipecat: \url{https://docs.pipecat.ai/}
  \item Clerk: \url{https://clerk.com/}
\end{itemize}

\vfill

\begin{center}
  \textcolor{secondary}{\small --- End of Document ---}
\end{center}

\end{document}
