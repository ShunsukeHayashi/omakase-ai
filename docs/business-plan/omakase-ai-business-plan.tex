\documentclass[a4paper,11pt]{article}

% Japanese support
\usepackage{luatexja}
\usepackage{luatexja-fontspec}
\setmainjfont{Hiragino Kaku Gothic ProN}

% Packages
\usepackage{geometry}
\usepackage{graphicx}
\usepackage{xcolor}
\usepackage{booktabs}
\usepackage{longtable}
\usepackage{array}
\usepackage{multirow}
\usepackage{enumitem}
\usepackage{tikz}
\usepackage{pgfplots}
\usepackage{fancyhdr}
\usepackage{hyperref}
\usepackage{tcolorbox}
\usepackage{titlesec}

% Page geometry
\geometry{top=25mm, bottom=25mm, left=20mm, right=20mm}

% Colors
\definecolor{primary}{RGB}{37, 99, 235}
\definecolor{secondary}{RGB}{59, 130, 246}
\definecolor{accent}{RGB}{16, 185, 129}
\definecolor{warning}{RGB}{245, 158, 11}
\definecolor{danger}{RGB}{239, 68, 68}
\definecolor{darkgray}{RGB}{55, 65, 81}
\definecolor{lightgray}{RGB}{243, 244, 246}

% Hyperref setup
\hypersetup{
    colorlinks=true,
    linkcolor=primary,
    urlcolor=secondary
}

% Header/Footer
\pagestyle{fancy}
\fancyhf{}
\fancyhead[L]{\small\textcolor{darkgray}{omakase.ai Business Plan}}
\fancyhead[R]{\small\textcolor{darkgray}{Confidential}}
\fancyfoot[C]{\thepage}
\renewcommand{\headrulewidth}{0.5pt}

% Section styling
\titleformat{\section}{\Large\bfseries\color{primary}}{\thesection}{1em}{}
\titleformat{\subsection}{\large\bfseries\color{secondary}}{\thesubsection}{1em}{}
\titleformat{\subsubsection}{\normalsize\bfseries\color{darkgray}}{\thesubsubsection}{1em}{}

% Custom boxes
\newtcolorbox{keypoint}{
    colback=lightgray,
    colframe=primary,
    fonttitle=\bfseries,
    title=Key Point
}

\newtcolorbox{highlight}{
    colback=accent!10,
    colframe=accent,
    fonttitle=\bfseries
}

\begin{document}

% Title Page
\begin{titlepage}
\centering
\vspace*{2cm}

{\Huge\bfseries\textcolor{primary}{omakase.ai}\par}
\vspace{0.5cm}
{\Large\textcolor{secondary}{事業計画書}\par}
\vspace{0.3cm}
{\large Business Plan Document\par}

\vspace{2cm}

{\large EC向け音声AIショッピングアシスタント\par}
{\large Voice AI Shopping Assistant for E-Commerce\par}

\vspace{3cm}

\begin{tcolorbox}[colback=lightgray, colframe=primary, width=0.8\textwidth]
\centering
\large
\textbf{ポジショニング・Go to Market戦略}\\
\textbf{開発プランニング・競合差別化戦略}
\end{tcolorbox}

\vspace{3cm}

{\large Version 1.0\par}
{\large 2025年12月6日\par}

\vfill

{\small Confidential - Internal Use Only\par}

\end{titlepage}

% Table of Contents
\tableofcontents
\newpage

%==============================================================================
\section{エグゼクティブサマリー}
%==============================================================================

\subsection{プロダクト概要}

omakase.aiは、ECサイト向けの\textbf{音声AIショッピングアシスタント}プラットフォームである。
Webウィジェットとして簡単に導入でき、顧客は音声で商品検索、質問、カート操作が可能となる。

\begin{keypoint}
\textbf{USP(独自の価値提案)}\\
「日本のEC事業者向けに、音声AIで購入転換率を\textbf{平均35\%向上}させる、EC特化型ショッピングアシスタント」
\end{keypoint}

\subsection{市場機会}

\begin{center}
\begin{tabular}{lrrr}
\toprule
\textbf{市場} & \textbf{2024年} & \textbf{2030年予測} & \textbf{CAGR} \\
\midrule
日本・音声コマース & \$2.47B & \$9.90B & \textbf{28.3\%} \\
グローバル・音声コマース & \$43.7B & \$252.5B & 19.9\% \\
グローバル・会話型AI & \$11.58B & \$41.39B & 23.7\% \\
\bottomrule
\end{tabular}
\end{center}

\begin{highlight}
\textbf{重要な発見}: 日本市場においてEC特化音声AIの競合は\textbf{0社}(ブルーオーシャン)
\end{highlight}

\subsection{3年成長シナリオ}

\begin{center}
\begin{tabular}{lrrrr}
\toprule
\textbf{指標} & \textbf{Year 1} & \textbf{Year 2} & \textbf{Year 3} \\
\midrule
顧客数 & 50社 & 200社 & 500社 \\
MRR & ¥13.5M & ¥56M & ¥167M \\
ARR & ¥97M & ¥560M & ¥1,874M \\
チャーン率 & 6\% & 3\% & 2\% \\
営業利益率 & -20\% & +24\% & +40\% \\
\bottomrule
\end{tabular}
\end{center}

\newpage
%==============================================================================
\section{市場分析}
%==============================================================================

\subsection{TAM/SAM/SOM分析}

\subsubsection{TAM(Total Addressable Market)}

\begin{itemize}
    \item \textbf{グローバル音声コマース市場}: 2030年 \$252.5B
    \item \textbf{グローバル会話型AI市場}: 2030年 \$41.39B
\end{itemize}

\subsubsection{SAM(Serviceable Available Market)}

\begin{itemize}
    \item \textbf{日本EC市場}: 年間約22兆円
    \item \textbf{日本音声コマース}: 2030年 \$9.90B(約1.5兆円)
    \item \textbf{EC向けAIツール市場}: 約2,000億円
\end{itemize}

\subsubsection{SOM(Serviceable Obtainable Market)}

\begin{itemize}
    \item \textbf{ターゲット}: 日本EC事業者 約3万サイト
    \item \textbf{3年後目標}: 500-1,000サイト(1.7-3.3\%シェア)
    \item \textbf{推定ARR}: ¥1.2B-2.4B
\end{itemize}

\subsection{市場トレンド}

\begin{enumerate}
    \item \textbf{Conversational Commerce成長}: 年率20\%以上で成長
    \item \textbf{Voice AI採用拡大}: スマートスピーカー普及率30\%超
    \item \textbf{EC事業者のAI投資増加}: CVR改善への投資意欲高
    \item \textbf{人手不足}: カスタマーサポート人材の確保困難
\end{enumerate}

\subsection{競合状況}

\begin{center}
\begin{tabular}{lccccc}
\toprule
\textbf{機能} & \textbf{omakase.ai} & \textbf{VAPI} & \textbf{Pipecat} & \textbf{Rep AI} & \textbf{Retell} \\
\midrule
音声対話 & $\checkmark$ & $\checkmark$ & $\checkmark$ & $\checkmark$ & $\checkmark$ \\
Webウィジェット & $\checkmark$ & - & - & $\checkmark$ & - \\
EC特化 & $\checkmark$ & - & - & $\checkmark$ & - \\
日本語 & $\checkmark$ & $\checkmark$ & $\checkmark$ & ? & $\checkmark$ \\
カート統合 & $\checkmark$ & - & - & $\checkmark$ & - \\
OSS & - & - & $\checkmark$ & - & - \\
\bottomrule
\end{tabular}
\end{center}

\newpage
%==============================================================================
\section{ポジショニング戦略}
%==============================================================================

\subsection{ポジショニングマップ}

omakase.aiは「\textbf{EC特化度}」と「\textbf{サービスレベル}」の2軸において、
市場の空白地帯を狙う。

\begin{center}
\begin{tikzpicture}
    % Axes
    \draw[->] (0,0) -- (10,0) node[right] {EC特化度};
    \draw[->] (0,0) -- (0,7) node[above] {サービスレベル};

    % Grid labels
    \node at (1,-0.5) {汎用};
    \node at (9,-0.5) {EC特化};
    \node at (-1.2,1) {セルフ};
    \node at (-1.2,6) {フル};

    % Competitors
    \fill[secondary] (2,2) circle (0.3) node[right, xshift=5pt] {VAPI};
    \fill[secondary] (3,1.5) circle (0.3) node[right, xshift=5pt] {Pipecat};
    \fill[secondary] (2.5,3) circle (0.3) node[right, xshift=5pt] {Retell};
    \fill[warning] (8,5.5) circle (0.3) node[right, xshift=5pt] {Rep AI};

    % omakase.ai
    \fill[accent] (8,4) circle (0.4) node[right, xshift=5pt] {\textbf{omakase.ai}};

    % White space annotation
    \draw[dashed, primary, thick] (6,2.5) rectangle (9,5);
    \node[primary] at (7.5,1.8) {White Space};
\end{tikzpicture}
\end{center}

\subsection{3つの差別化軸}

\begin{enumerate}
    \item \textbf{EC販売最適化}: 汎用AIではなく「ECで売る」に特化
    \item \textbf{日本市場特化}: 日本語敬語、日本の商習慣、コンプライアンス対応
    \item \textbf{即導入可能}: 5分実装、エンジニア不要
\end{enumerate}

\subsection{価値提案(Value Proposition)}

\begin{center}
\begin{tabular}{ll}
\toprule
\textbf{顧客の課題} & \textbf{omakase.aiの解決策} \\
\midrule
CVR向上の限界(2.5\%の壁) & 音声AIで+35\%のCVR改善 \\
カスタマーサポートコスト増 & 24時間自動対応で人件費削減 \\
カゴ落ち率の高さ(70\%) & リアルタイム音声アシストで回収 \\
技術導入のハードル & 5分で導入、コード不要 \\
\bottomrule
\end{tabular}
\end{center}

\newpage
%==============================================================================
\section{Go-to-Market戦略}
%==============================================================================

\subsection{フェーズ別戦略}

\subsubsection{Phase 1: MVP/PMF検証(3-4ヶ月)}

\begin{itemize}
    \item \textbf{目標}: ベータ顧客5社でPMF検証
    \item \textbf{ターゲット}: Shopify利用のD2Cブランド
    \item \textbf{KPI}: CVR +15\%の実証
    \item \textbf{チャネル}: 直接アプローチ、紹介
\end{itemize}

\subsubsection{Phase 2: 初期顧客獲得(3-4ヶ月)}

\begin{itemize}
    \item \textbf{目標}: 有料顧客30社、MRR ¥1.5M
    \item \textbf{ターゲット}: 中規模EC(年商1億-50億円)
    \item \textbf{チャネル}: コンテンツマーケ30\%、パートナー25\%、有料広告20\%
\end{itemize}

\subsubsection{Phase 3: スケール(6ヶ月〜)}

\begin{itemize}
    \item \textbf{目標}: 150社、MRR ¥9M
    \item \textbf{ターゲット}: BASE/STORES拡大、エンタープライズ開拓
    \item \textbf{チャネル}: Shopifyアプリストア、システムインテグレーター
\end{itemize}

\subsection{プライシング戦略}

\begin{center}
\begin{tabular}{lrrll}
\toprule
\textbf{プラン} & \textbf{月額} & \textbf{対話数} & \textbf{対象} & \textbf{ROI} \\
\midrule
Starter & ¥30,000 & 1,000 & 小規模EC & 54x \\
Growth & ¥80,000 & 3,000 & 中規模EC & 27x \\
Business & ¥200,000 & 10,000 & 大規模EC & 45x \\
Enterprise & 要相談 & 無制限 & エンタープライズ & - \\
\bottomrule
\end{tabular}
\end{center}

\subsection{チャネル戦略}

\begin{center}
\begin{tikzpicture}
    \pie[
        text=legend,
        radius=2.5,
        color={primary!80, secondary!80, accent!80, warning!80, danger!60}
    ]{
        30/コンテンツマーケティング,
        25/パートナー経由,
        20/有料広告,
        15/イベント・展示会,
        10/紹介プログラム
    }
\end{tikzpicture}
\end{center}

\newpage
%==============================================================================
\section{開発ロードマップ}
%==============================================================================

\subsection{フェーズ別開発計画}

\subsubsection{Phase 1: MVP(3-4ヶ月)}

\begin{itemize}
    \item 音声AIアシスタント基盤
    \item Widget UI
    \item Shopify連携
    \item 管理ダッシュボード(基本)
    \item EC特化プロンプト設計
\end{itemize}

\textbf{技術スタック}: VAPI, Daily.co WebRTC, Express.js, React

\subsubsection{Phase 2: Growth(3-4ヶ月)}

\begin{itemize}
    \item PostgreSQL導入(In-Memoryから移行)
    \item 分析ダッシュボード強化
    \item 業種別プロンプトテンプレート
    \item A/Bテスト基盤
    \item BASE/STORES連携
\end{itemize}

\subsubsection{Phase 3: Scale(6ヶ月)}

\begin{itemize}
    \item Pipecat検証・移行(コスト37\%削減)
    \item パフォーマンス最適化
    \item マルチテナント対応
    \item LINE連携
\end{itemize}

\subsubsection{Phase 4: Platform(6ヶ月)}

\begin{itemize}
    \item 自社LLM Fine-tuning
    \item SDK公開
    \item マーケットプレイス構築
\end{itemize}

\subsection{技術KPI}

\begin{center}
\begin{tabular}{lcccc}
\toprule
\textbf{指標} & \textbf{Phase 1} & \textbf{Phase 2} & \textbf{Phase 3} & \textbf{Phase 4} \\
\midrule
稼働率 & 99.5\% & 99.7\% & 99.9\% & 99.99\% \\
レスポンス時間 & <5秒 & <4秒 & <3秒 & <2秒 \\
同時接続数 & 100 & 500 & 2,000 & 10,000 \\
\bottomrule
\end{tabular}
\end{center}

\newpage
%==============================================================================
\section{競合差別化戦略}
%==============================================================================

\subsection{持続可能な競争優位の構築}

\subsubsection{短期的差別化(0-6ヶ月)}

\begin{itemize}
    \item \textbf{5分導入Widget}: 導入障壁の排除
    \item \textbf{成果報酬型価格}: リスクリバーサル
    \item \textbf{日本語最適化}: 敬語、方言対応
\end{itemize}

\subsubsection{中期的差別化(6-12ヶ月)}

\begin{itemize}
    \item \textbf{LINE連携}: 日本固有チャネル対応
    \item \textbf{物流API統合}: ヤマト/佐川の配送日時指定
    \item \textbf{業界特化プロンプトライブラリ}: 10業種対応
\end{itemize}

\subsubsection{長期的差別化(1-2年)}

\begin{itemize}
    \item \textbf{会話×購買データ}: 独自モデル開発
    \item \textbf{ECプラットフォームOEM}: B2B提供
    \item \textbf{B2B市場展開}: 高単価エンタープライズ
\end{itemize}

\subsection{4つの防御壁}

\begin{center}
\begin{tabular}{ll}
\toprule
\textbf{防御メカニズム} & \textbf{具体施策} \\
\midrule
ネットワーク効果 & 共有インテントライブラリ(導入店舗増=AI精度向上) \\
データモート & 会話→購買データの蓄積(教師データ資産化) \\
ブランド & 日本品質、ISMAP/Pマーク、日本語サポート \\
スイッチングコスト & オペレーション統合(解約=人員補充必要) \\
\bottomrule
\end{tabular}
\end{center}

\subsection{VAPI依存リスク対策}

\begin{center}
\begin{tabular}{lll}
\toprule
\textbf{期間} & \textbf{戦略} & \textbf{具体策} \\
\midrule
短期 & 関係強化 & Japan Gateway戦略(日本市場パートナー) \\
中期 & マルチベンダー & 抽象化レイヤー(Voice Router)で複数対応 \\
長期 & ハイブリッド & 自社モデル併用、買収/提携検討 \\
\bottomrule
\end{tabular}
\end{center}

\newpage
%==============================================================================
\section{ターゲットペルソナ}
%==============================================================================

\subsection{B2B 導入決定者}

\subsubsection{ペルソナ1: ECマーケティング責任者(最優先)}

\begin{itemize}
    \item \textbf{属性}: 35-45歳、年商10-50億円EC企業
    \item \textbf{課題}: CVR向上の限界、広告費高騰
    \item \textbf{ゴール}: 売上20\%増、差別化施策
    \item \textbf{決定要因}: ROI実証、導入の簡単さ
    \item \textbf{支払意思額}: ¥20-50万円/月
\end{itemize}

\subsubsection{ペルソナ2: EC技術責任者}

\begin{itemize}
    \item \textbf{属性}: 30-40歳、エンジニアバックグラウンド
    \item \textbf{課題}: 人手不足、新技術評価の時間不足
    \item \textbf{ゴール}: 安定運用、セキュリティ確保
    \item \textbf{決定要因}: 技術仕様、API品質、サポート体制
\end{itemize}

\subsubsection{ペルソナ3: EC事業経営者}

\begin{itemize}
    \item \textbf{属性}: 40-55歳、D2C/小売経営者
    \item \textbf{課題}: 競合との差別化、人件費増
    \item \textbf{ゴール}: 利益率改善、ブランド価値向上
    \item \textbf{決定要因}: 投資対効果、市場でのポジショニング
\end{itemize}

\subsection{B2C エンドユーザー}

\subsubsection{ペルソナ4: デジタルネイティブ主婦}

\begin{itemize}
    \item \textbf{属性}: 30-40歳、子育て中
    \item \textbf{利用シーン}: 料理中、育児中のハンズフリー買い物
    \item \textbf{価値}: 時短、便利さ
\end{itemize}

\subsubsection{ペルソナ5: ガジェット好き若手社会人}

\begin{itemize}
    \item \textbf{属性}: 25-35歳、新しいもの好き
    \item \textbf{利用シーン}: 通勤中、就寝前
    \item \textbf{価値}: 新体験、効率性
\end{itemize}

\newpage
%==============================================================================
\section{収益モデル}
%==============================================================================

\subsection{ハイブリッド収益モデル}

\begin{center}
\begin{tikzpicture}
    \pie[
        text=legend,
        radius=2.5,
        color={primary!90, secondary!80, accent!70, warning!60}
    ]{
        70/SaaSサブスク,
        15/従量課金(超過分),
        10/カスタマイズ開発,
        5/成果報酬
    }
\end{tikzpicture}
\end{center}

\subsection{Unit Economics}

\begin{center}
\begin{tabular}{lrl}
\toprule
\textbf{指標} & \textbf{数値} & \textbf{評価} \\
\midrule
LTV & ¥2,000,000 & - \\
CAC & ¥120,000 & - \\
LTV/CAC & 16.7x & Excellent(5x以上が優秀) \\
Payback期間 & 2.0ヶ月 & Very Good(12ヶ月以内が理想) \\
Rule of 40 & 500\% & Excellent(40\%以上が健全) \\
\bottomrule
\end{tabular}
\end{center}

\subsection{損益分岐点分析}

\begin{itemize}
    \item \textbf{損益分岐点}: Year 2 Q2で黒字転換
    \item \textbf{必要顧客数}: 約80社
    \item \textbf{必要MRR}: ¥8.7M
\end{itemize}

\newpage
%==============================================================================
\section{実行計画}
%==============================================================================

\subsection{直近3ヶ月のアクションプラン}

\subsubsection{Month 1: 準備}

\begin{itemize}
    \item GTM戦略の社内共有
    \item マーケティング責任者採用
    \item ベータ募集LP作成
    \item CRM/MAツール選定
\end{itemize}

\subsubsection{Month 2: ベータ開始}

\begin{itemize}
    \item ベータ顧客5社確定
    \item オンボーディング実施
    \item A/Bテスト開始
\end{itemize}

\subsubsection{Month 3: PMF検証}

\begin{itemize}
    \item データ分析
    \item PMFスコアカード作成
    \item Phase 2移行判断
\end{itemize}

\subsection{今すぐ実行すべきTOP 10}

\begin{enumerate}
    \item 抽象化レイヤー(Voice Router)設計
    \item カゴ落ち×音声の実証実験
    \item 5分導入動画作成
    \item LINE連携プロトタイプ
    \item 商標・ドメイン防衛
    \item カスタマーサクセス採用
    \item B2B企業ヒアリング
    \item VAPIパートナーシップ交渉
    \item 日本固有語彙辞書整備
    \item セキュリティチェックシート作成
\end{enumerate}

\newpage
%==============================================================================
\section{リスクと対策}
%==============================================================================

\subsection{主要リスク}

\begin{center}
\begin{tabular}{lll}
\toprule
\textbf{リスク} & \textbf{影響度} & \textbf{対策} \\
\midrule
VAPIのEC垂直統合 & 高 & 日本ローカル対応で差別化 \\
Rep AI日本参入 & 中 & LINE連携、B2B展開で差別化 \\
大手EC内製化 & 高 & 公式パートナー化交渉 \\
技術的参入障壁低 & 中 & データモート構築、先行者優位確立 \\
\bottomrule
\end{tabular}
\end{center}

\subsection{対策の詳細}

\subsubsection{vs VAPI垂直統合}

日本市場特有の要素(配送連携、決済方法、敬語対応)で防御壁を構築。
VAPIがEC展開しても、日本ローカライゼーションでは常に優位。

\subsubsection{vs 大手ECプラットフォーム内製化}

\begin{itemize}
    \item 公式パートナー化交渉(Shopify Japan、BASE等)
    \item OEM提供モデルへの転換
    \item ニッチセグメント(高齢者向け、B2B等)への特化
\end{itemize}

%==============================================================================
\section{結論}
%==============================================================================

\begin{keypoint}
\textbf{omakase.aiの勝機}

AI技術の凄さではなく、\\
「\textbf{日本の商習慣にAIをフィットさせるインテグレーション力}」

VAPIはエンジン、omakase.aiは\textbf{日本の道路事情に完璧に合わせた高級車}
\end{keypoint}

\subsection{戦略の核心}

\begin{enumerate}
    \item \textbf{日本市場ブルーオーシャン}: EC特化音声AI競合0社
    \item \textbf{垂直特化}: 汎用AIと差別化、EC販売に最適化
    \item \textbf{先行者優位}: 2025-2026年が確立の最後のチャンス
    \item \textbf{データモート}: 会話×購買データで模倣困難な資産構築
\end{enumerate}

\vspace{1cm}

\begin{center}
\textbf{「その声が、売上になる。」}
\end{center}

\vfill

\begin{center}
\textcolor{darkgray}{\small --- End of Document ---}
\end{center}

\end{document}
